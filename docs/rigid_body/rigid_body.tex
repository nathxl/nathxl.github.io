\documentclass[12pt]{article}
\usepackage{amsmath}
\usepackage{amsfonts}
\usepackage{geometry}
\usepackage{multicol}
\geometry{margin=0.75in}
\setlength{\parindent}{0pt}
\begin{document}

\section{2D rigid body}

To model the 2d rigid body, we will start from a 2D polygon made of n points.

Each point will have a position and a mass.

We can get the total mass of the rigid body by summing the mass of its points :
$$ m = \sum m_i $$

We can get the center of mass of the rigid body as such :
$$ \bar{R} = \frac{\sum m_i \bar{r}_i}{\sum m_i} $$

We can get the moment of inertia of the rigid body as such :
$$ I = \sum m_i r_i^2$$

Now we can give initial conditions to the rigid body, namely :

\begin{itemize}
\item $\bar{r}_0$ : initial position
\item $\bar{v}_0$ : initial velocity
\item $\theta_0$ : initial angle
\item $\omega_0$ : initial angular velocity
\end{itemize}

We have the following relationships :



\begin{multicols}{2}
in the continous domain
$$ \bar{a} = \frac{d \bar{v}}{dt}$$
$$ \bar{v} = \frac{d \bar{r}}{dt}$$
$$ \alpha = \frac{d \omega}{dt}$$
$$ \omega = \frac{d \theta}{dt}$$
\columnbreak

in the discrete domain
$$ \bar{a} = \frac{\Delta \bar{v}}{\Delta t}$$
$$ \bar{v} = \frac{\Delta \bar{r}}{\Delta t}$$
$$ \alpha = \frac{\Delta \omega}{\Delta t}$$
$$ \omega = \frac{\Delta \theta}{\Delta t}$$
\end{multicols}

Each cycle of the physics loop, we can apply forces on the rigid body :

$$ \bar{F} = m \bar{a} $$

If the force is applied elsewhere than the center of mass, it will also provoque a torque on the rigid body :

$$ \bar{\tau} = \bar{r} \times \bar{F} $$
$$ \bar{\tau} = I \bar{\alpha} $$

\newpage

Each cycle of the physics loop, we can know the acceleration on the rigid body, thus its velocity, thus its position as such :

$$ \bar{a} = \frac{\sum \bar{F}}{m} $$

$$ \bar{v}_n = \bar{v}_{n - 1} + \bar{a} \Delta t $$

$$ \bar{r}_n = \bar{r}_{n - 1} + \bar{v}_n \Delta t $$

Each cycle of the physics loop, we can know the angular acceleration on the rigid body, thus its angular velocity, thus its angular position as such :


$$ \bar{\alpha} = \frac{\sum \bar{\tau}}{I} $$
$$ \bar{\omega}_n = \bar{\omega}_{n-1} + \bar{\alpha} \Delta t $$
$$ \bar{\theta}_n = \bar{\theta}_{n-1} + \bar{\omega}_n \Delta t $$

\end{document}